\section{Einleitung}
Der Simplex-Downhill-Algorithmus ist eine Methode zur Optimierung nicht-linearer n-dimensionaler Funktionen. \\
Dieses Verfahren wurde von den britischen Statistikern John Nelder und Roger Mead entwickelt und gilt vor allem bei Problemen mit geringem rechnerischen Aufwand als die schnellstmögliche Methode. Ein Ein klarer Vorteil dieser Methode ist, dass sie auf Probleme angewendet werden kann, bei denen  die Ableitungen nicht bekannt sind oder gar nicht existieren. Die Methode kann hilfreich sein, um (sinnvolle) Versuchspunkte zu ermitteln, welche dann in eine Simulation eingespeist werden. Dies hat den Vorteil, dass eine teure (sprich zeit- und rechenaufwendige) Simulation nicht einfach sinnlose Datenmengen rechnet, die man schlussendlich verwerfen muss, weil sie physikalisch keinen Sinn ergeben.  
