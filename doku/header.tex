%%%%%%%%%%%%%%%%%%%%%%%%%%%%%%%%%%%%%%%%%%%%%%%%%%
%% Allgemeiner Header mit scrreprt				%%
%% Autor: Stefan Zollinger						%%
%% Quellen:	-scrguide							%%		
%%			-www.dante.de						%%
%%%%%%%%%%%%%%%%%%%%%%%%%%%%%%%%%%%%%%%%%%%%%%%%%%

%-------------------------------------------------
%Dokumentklasse und allgemeine Pakete
%-------------------------------------------------
\documentclass[
		11pt, 
		bibliography=totocnumbered,		    %Literaturverzeichnis im Inhaltsverzeichnis 
	  listof=totocnumbered,							%Tab. und Abb.-verzeichnis nummeriert	
		final,												%f�r endversion (draft -> ohne bilder und links, 
		                              %anzeige von Fehlern)
		parskip=half,									%Halbe Zeile Abstand zwischen Abs�tzen.
		twoside=false									%zweiseitig
		]{scrreprt} 	

\usepackage{scrhack}					%beseitigt warnung wegen float+koma inkompatibilit�t
\usepackage[T1]{fontenc}			%umlaute als eigene Zeichen
\usepackage[latin1]{inputenc}	%umlaute erkennen
\usepackage[ngerman]{babel}		%silbentrennung englische Rechtschreibung
\usepackage{babelbib}					%f�r deutsches literaturverzeichnis
%\usepackage{scrpage2}					%pagestyle
\usepackage{lastpage}					%f�r Referenzen auf letzte Seite
\usepackage{textcomp}					%einige Sonderzeichen
\usepackage{graphicx}	        %grafiken in jpeg,png
\usepackage{epstopdf}		    	%f�r .eps vektorgrafiken
\usepackage{color}						%ben�tigt f�r farbeinstellungen
\usepackage{array}				    %erweiterte Optionen in tabellen (Bsp. ausrichtung innerhalb felder)
\usepackage{subfigure}				%um 2 bilder nebeneinander anzuzeigen
\usepackage{amsmath} 					%mathematischen Textsatz.
\usepackage{amssymb}					%	-Erweiterte math. Sonderzeichen
\usepackage{dsfont}						%	-Mengen			
\usepackage{paralist} 				%kompakte Aufz�hlung: \begin{compactitem}
\usepackage{float}						%f�r H option bei floats(figure,table)
\usepackage{lmodern}					%moderne Schrift
\usepackage{url}				   	 	%korrekte darstellung von urls (mit verlinkung)
\usepackage{changepage}       %f�r adjustwidth auf titelseite
\usepackage{fancyhdr}					%f�r \pagestyle{fancy} -> kopf/fusszeile �ber Textk�rper hinaus
\usepackage{acronym}					%abk�rzungsverzeichnis, Anwendung: \ac{KDE}
\usepackage{cellspace}				%f�r S-operator in Tabellen -> Formeln ber�hren R�nder nicht
\usepackage{booktabs}					%Tabellen: \toprule \midrule \bottomrule
\usepackage{mparhack}         %Repariert falsch gesetzte marginalien am Seitenanfang
\usepackage{setspace}
\usepackage{placeins}
\usepackage{tikz}
\usetikzlibrary{arrows,shapes,snakes}

%-------------------------------------------------
% formatierte Listings
%-------------------------------------------------	
\definecolor{commentColor}{rgb}{0.3,0.3,0.3}	%Grau f�r kommentare in Listings
\usepackage{listings}
\lstset{ %
language=Matlab,                % choose the language of the code
basicstyle=\small \ttfamily,    % font style
numbers=none,                   % where to put the line-numbers
numberstyle=\small \ttfamily,      % the size of the fonts that are used for the line-numbers
commentstyle=\color{commentColor},				%kommentare
stepnumber=1,                   % the step between two line-numbers. If it's 1 each line will be numbered
numbersep=5pt,                  % how far the line-numbers are from the code�
xleftmargin=15pt,               % linker rand
xrightmargin=15pt,               % rechter rand
backgroundcolor=\color{white},  % choose the background color. requires \usepackage{color}
showspaces=false,               % show spaces adding particular underscores
showstringspaces=false,         % underline spaces within strings
showtabs=false,                 % show tabs within strings adding particular underscores
frame=lines,  	                  % frame: none|leftline|topline|bottomline|lines|single|shadowboxi
tabsize=2,	                    % sets default tabsize to 2 spaces
captionpos=t,                   % sets the caption-position t/b
breaklines=true,                % sets automatic line breaking
breakatwhitespace=false,        % sets if automatic breaks should only happen at whitespace
escapeinside={\%*}{*)}          % if you want to add a comment within your code
}


%-------------------------------------------------
% Abk�rzungsverzeichnis implementieren
%-------------------------------------------------									
\usepackage{nomencl} 				
\renewcommand{\nomname}{List of abbreviations}
\makenomenclature					%abk.verz. umbenennen

%kommando zum einfachen einstellen der Schrift
% Bsp. \changefont{lmss}{sbc}{n}  %latin modern bold condensed Schriftart der Titel
%      \changefont{lmss}{m}{n}		%latin modern sans serif
\newcommand{\changefont}[3]{\fontfamily{#1}\fontseries{#2}\fontshape{#3}\selectfont}
  
%-------------------------------------------------
% Marginalien/Seitenr�nder
%-------------------------------------------------
\newcommand{\marg}[1]{\marginpar{\raggedright \changefont{lmss}{m}{n} #1}}	
%marginalien linksb�ndig mit \marg{text der marginalie}

%\usepackage[
%			%includemp,				%marginalien in Textk�rper einbeziehen
%			%includeall,
%			%showframe,				%zeigt rahmen zum debuggen		
%			marginparwidth=30mm, 	%breite der marginalien
%			marginparsep=5mm,		%abstand marginalien - text
%			reversemarginpar,		%marginalien links statt rechts
%			left=45mm,				%abstand von Seitenraendern
%			right=25mm,				%
%			top=30mm,
%			bottom=30mm,
%			twoside
%			]{geometry}
			
\usepackage[
			%includemp,				%marginalien in Textk�rper einbeziehen
			%includeall,
			%showframe,				%zeigt rahmen zum debuggen		
			marginparwidth=30mm, 	%breite der marginalien
			marginparsep=5mm,		%abstand marginalien - text
			%reversemarginpar,		%marginalien links statt rechts
			left=25mm,				%abstand von Seitenraendern
			right=45mm,				%
			top=20mm,
			bottom=20mm,
			oneside
			]{geometry}
			
%-------------------------------------------------
%Kopf-/Fusszeile
%-------------------------------------------------
%\pagestyle{scrheadings}
%\clearscrheadfoot															%voreinstellungen l�schen
%\automark[chapter]{chapter}										%Kapitelangabe 
%\ihead{\changefont{lmss}{sbc}{n} \headmark}		%kopfzeile
%\ifoot[																				%fusszeile
%        \changefont{lmss}{m}{n}{ \thepage  /\pageref{LastPage}}]
%        {\changefont{lmss}{m}{n}{ \thepage  /\pageref{LastPage}}}	
        
        
        
%%% Kopf und Fusszeilen
\pagestyle{fancy}					%stil der kopf/fusszeilen 
\renewcommand{\chaptermark}[1]{\markboth{\thechapter.\ #1}{}}	
%damit Kapitelangaben im Header kleingeschrieben werden
\fancyhead{} 						% clear all header fields
\fancyhead[RE,LO]{ \changefont{lmss}{sbc}{n} \leftmark }
\fancyhead[C]{ }
\fancyhead[LE,RO]{ \changefont{lmss}{m}{n} \thepage/\pageref{LastPage} }
%\fancyhead[LO]{\changefont{lmss}{m}{n} \thepage/\pageref{LastPage} \hspace{2.5cm}
%							 \changefont{lmss}{sbc}{n} \leftmark }
\fancyfoot{} 						% clear all footer fields
\fancyfoot[LE,RO]{ }
\fancyfoot[C]{}
\fancyfoot[RE,LO]{  }
\renewcommand{\headrulewidth}{0.5pt}
\renewcommand{\footrulewidth}{0pt}
\fancyhfoffset[LE,RO]{35mm} 			%header und footer nach links verlängern (über marginalien)
\fancypagestyle{plain}{				% plain redefinieren, damit wirklich alle seiten im gleichen stil sind (ausser titlepage)
	\pagestyle{fancy}
}

%-------------------------------------------------
%PDF-Optionen
%-------------------------------------------------
\usepackage[%
	pdftitle={SA},%                                   Titel des PDF Dokuments.
	pdfauthor={Autoren},%                                Autor des PDF Dokuments.
	pdfsubject={Thema},%                                 Thema des PDF Dokuments.
	pdfcreator={LaTeX with hyperref and KOMA-Script},%   Erzeuger des PDF Dokuments.
	pdfkeywords={Schuelsselwrter},%                        auch fr PDF Dokumente indexiert.)
	pdfpagemode=UseOutlines,%                            Inhaltsverzeichnis anzeigen beim ffnen
	pdfdisplaydoctitle=false,%                            Dokumenttitel statt Dateiname anzeigen.
	pdflang=de%                                          Sprache des Dokuments.
]{hyperref}

\definecolor{LinkColor}{rgb}{0,0,0}	%{0,0,0.4} -> Blaue Links im PDF Dokument.
\hypersetup{%
	colorlinks=true,%        Aktivieren von farbigen Links im Dokument (keine Rahmen)
	linkcolor=LinkColor,%    Farbe festlegen.
	citecolor=LinkColor,%    Farbe festlegen.
	filecolor=LinkColor,%    Farbe festlegen.
	menucolor=LinkColor,%    Farbe festlegen.
	urlcolor=LinkColor,%     Farbe von URL's im Dokument.
	bookmarksnumbered=true%  berschriftsnummerierung im PDF Inhalt anzeigen.
}

% bessere float plazierung
\renewcommand{\textfraction}{0.1}
\renewcommand{\topfraction}{0.9}
\renewcommand{\bottomfraction}{0.9}
\renewcommand{\floatpagefraction}{0.35}
\setcounter{totalnumber}{5}

\let\origsection\section								%\section sichern als \origsection
\renewcommand{\section}{\FloatBarrier \origsection}		%\section �berschreiben 
														%\FloatBarrier: verhindert, dass Bilder oder Tab. hier weiterrutschen
